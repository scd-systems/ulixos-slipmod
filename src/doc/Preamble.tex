%%%
%%% Preamble
%%%

%% Design Wahl
\documentclass[11pt,a4paper,oneside]{article}

% Umlaute Fonts
\usepackage[latin9]{inputenc}
\usepackage[ngerman]{babel}
% \usepackage[T1]{fontenc}
\usepackage{graphicx}
\usepackage{listings}
\lstset{basicstyle=\footnotesize\ttfamily,breaklines=true}
%\lstset{framextopmargin=50pt,frame=bottomline}
% Courier f. Listings
\usepackage{courier}
% F�r Griechisch (Micro) Zeichen
\usepackage{textcomp} 
\lstloadlanguages{R}
% Runde Klammern
%\usepackage[square]{natbib}
% Viereckige Klammern
\usepackage[square,sort,comma,numbers]{natbib}
% F�r Bildtextuml�ufe option [H] verwenden
\usepackage{float}
\usepackage{subfigure}
% Table: Multirow
\usepackage{multirow}
% Rotation 
\usepackage{rotating}
% NoWeb
\usepackage{noweb}
% PDFs Einbinden
\usepackage{pdfpages}
\usepackage{hyperref}

\definecolor{LinkColor}{rgb}{0.18,0,0.44}
\hypersetup{
  pdfauthor={C. Dornig},
  pdftitle={SLIP Module for Ulix OS},
  colorlinks=true,
  urlcolor=blue,
  linkcolor=LinkColor
}
\noweboptions{smallcode,english,externalindex}
\usepackage{url,graphicx,boxedminipage,a4wide}
\newcommand\hexaddr[1]{{\small\tt 0x#1}}

\newcommand{\eg}{e.\,g.}

% for indexing (definitions from Felix)
% plain index, additional term to be added (does not appear in text)
\newcommand{\pindex}[1]{\index{#1}}
% like plain index, term is sorted according to first argument but
% typeset as second argument (example: \pcindex{TeX}{\TeX{}})
\newcommand{\pcindex}[2]{\index{#1@#2}}
% verbatim index, term appears in index and in text as is
\newcommand{\vindex}[1]{#1\index{#1}}
%
% indicate where work needs to be done
\newenvironment{work}{\noindent\begin{boxedminipage}{\textwidth}}{\end{boxedminipage}}

\newcommand{\shellcmd}[1]{{\tt #1}}
\newcommand{\codesymbol}{*}
\newcommand{\codesection}[1]{\section{#1\codesymbol{}}}

% give normal text in math formulas
\newcommand{\text}[1]{\mbox{#1}}

% Zitate: ,,...�� mit \anf
\newcommand{\DQ}[1]{\glqq{}#1\grqq{}}

% NoWEB: Remove to much whitespace in the buttom of the pages
\def\nwendcode{\endtrivlist \endgroup}
\let\nwdocspar=\par

% Abk�rzungsverzeichnis Modul
\usepackage{nomencl}
\let\Abbrev\nomenclature
\renewcommand{\nomname}{Abk�rzungsverzeichnis} 
% Den Abstand zwischen Abk�rzung und Erkl�rung mit Punkten auff�llen
\setlength{\nomlabelwidth}{.50\hsize}
\renewcommand{\nomlabel}[1]{#1 \dotfill}
\setlength{\nomitemsep}{-\parsep}

%%%%

\begin{document}
\setcounter{tocdepth}{4}

\input{Titelpage.tex}

%Inhaltsverzeichnis
%\PrintContents
\tableofcontents 
\pagenumbering{Roman}

\newpage
\section{Verzeichnisse}
\subsection{Abbildungsverzeichnis}
\listoffigures

\newpage
\subsection{Abk�rzungsverzeichnis}
\makenomenclature
\printnomenclature

%\newpage
%\subsection{Tabellenverzeichnis}
%\listoftables

\newpage
\subsection{Listingverzeichnis}
\lstlistoflistings

%%% Text
\pagenumbering{arabic}

\newpage
\input{slip-mod}

\newpage
\newpage
\section{Literatur}
\begin{thebibliography}{------}

\bibitem[AGI14]{agi14}
	{Agile Manifesto - Manifesto for agile software development},
	URL: \url{http://www.agilemanifesto.org/iso/de/}, Zugriff: Mai 2014

\bibitem[ASAAS04]{asaas04}
	{Liakot Ali, Roslina Sidek, Ishak Aris, Alauddin Mohd. Ali, Bambang Sunaryo Suparjo},
	{\em Design of a micro-UART for SoC application}
	Computers and Electrical Engineering, Volume 30, 2004, S. 257-268

\bibitem[BPH10]{bph10}
	{Richard Baskerville, Jan Pries-Heje},
	{\em Erkl�rende Designtheorie}
	Wirtschaftsinformatik, 5, 2010, S. 259-271

\bibitem[COC01]{coc01}
	{Andy Cockburn},
	{\em Supporting tailorable program visualisation through literate programming and fisheye views}
	Information and Software Technology, 43, 2001, S. 745-758	

\bibitem[COO98]{coo98}
	{Barry M.Cook},
	{\em IEEE 1355 data-strobe links: ATM speed at RS232 cost}
	Microprocessors and Microsystems, 21, 1998, S. 421-428 

\bibitem[DIA14]{dia14}
	{Dia - GTK+ based diagram creation program for GNU/Linux},
	URL: \url{https://wiki.gnome.org/Apps/Dia}, Zugriff: M�rz 2014

\bibitem[DNBM12]{dnbm12}
	{Torgeir Dings�yr, Sridhar Nerur, VenuGopal Balijepally, Nils Brede Moe},
	{\em A decade of agile methodologies: Towards explaining agile software development}
	The Journal of Systems and Software, 85, 2012, S. 1213 - 1221

%\bibitem[E�E13]{ese13}
%	{Hans-Georg E�er},
%	{\em Design, Implementation, and Evaluation of the ULIX-i386 Teaching Operating System}
%	University of Erlangen, 2013, S. 1.438

\bibitem[EKRSV05]{ekrsv05}
	{Erich Ehses, Lutz K�hler, Petra Riemer, Horst Stenzel, Frank Victor},
	{\em Betriebssysteme - Ein Lehrbuch mit �bungen zur Systemprogrammierung in UNIX/Linux}
	Pearson Studium, 2005

\bibitem[FBSD14]{fbsd14}
	{FreeBSD},
	URL: \url{www.freebsd.org}, Zugriff: M�rz 2014

\bibitem[FPP14]{fpp14}
	{FreeBSD - Packages and Ports},
	URL: \url{http://www.freebsd.org/doc/en\_US.ISO8859-1/books/handbook/ports.html}, Zugriff: Apr. 2014

\bibitem[FTS14]{fts14}
	{FreeBSD - Quellcode TCP/IP Stack},
	URL: \url{http://svnweb.freebsd.org/base/head/sys/netinet/}, Zugriff: Apr. 2014

\bibitem[FUN14]{fun14}
	{FunnelWEB},
	URL: \url{http://www.ross.net/funnelweb/}, Zugriff: April 2014

\bibitem[GCC14]{gcc14},
	{GNU Compiler Collection}
	URL: \url{http://gcc.gnu.org/}, Zugriff: M�rz 2014

\bibitem[ISO94]{iso94}
	{ISO/IEC},
	{\em Open System Interconnect - Basis Reference Model}
	ISO/IEC 7498:1:1994(E), 1994

\bibitem[KJKAA04]{kjkaa04}
	{Ajan Daniel Kutty, Bestin Jose, Ciju Rajan K, Daise Antony, Linto Antony},
	{\em Serial Line IP Implementation for Linux Kernel TCP/IP Stack}
	URL: \url{www.cse.iitb.ac.in/~bestin/pdfs/slip.pdf}, Zugriff: M�rz 2014

\bibitem[KNU83]{knu83}
	{Donald E. Knuth},
	{\em Literate Programming}
	Computers Journal, 1983, S. 1-15

\bibitem[LAT14]{lat14}
	{\TeX, \LaTeX},
	URL: \url{www.latex-project.org}, Zugriff: M�rz 2014

\bibitem[LIN14]{lin14}
	{Linux Kernel},
	URL: \url{http://www.kernel.org}, Zugriff: M�rz 2014

\bibitem[LWIP14]{lwip14}
	{lwIP - A Lightweight TCP/IP stack},
	URL: \url{http://savannah.nongnu.org/projects/lwip/}, Zugriff: M�rz 2014

\bibitem[MJRRH03]{mjrrh03}
	{P. M�h�nen, J. Riihij�rvi, O. Raivio, Pertti Huuskonen},
	{\em NanoIP: The Zen of Embedded Networking}
	URL: \url{IEEE International Conference on Communications, 2003, Volume 2, S. 1218-1222}

\bibitem[NO14]{no14}
	{Noweb - A Simple, Extensible Tool for Literate Programming},
	URL: \url{http://www.cs.tufts.edu/~nr/noweb/}, Zugriff: April 2014

\bibitem[NSC95]{nsc95}
	{National Semiconductor Corporation},
	{\em PC16550D Universal Asynchronous Receiver Transmitter with FIFOs }
	RRD-B30M75 Printed in USA

\bibitem[NU14]{nu14}
	{The nuweb system for Literate Programming},
	URL: \url{http://nuweb.sourceforge.net/}, Zugriff: April 2014

\bibitem[OBSD14]{obsd14}
	{OpenBSD},
	URL: \url{www.openbsd.org}, Zugriff: Mai 2014
	
\bibitem[PEP91]{pep91}
	{Peter Pepper},
	{\em Literate Program Derivation: A Case Study}
	Lecture Notes in Computer Science, Volume 544, 1991, S. 101-124

\bibitem[QEMU14]{qemu14},
	{QEMU - generic and open source machine emulator and virtualizer}
	URL: \url{http://wiki.qemu.org/Main_Page}, Zugriff: April 2014

\bibitem[RISA09]{risa09}
	{David F. Rico, Hasan H. Sayani},
	{\em Use of agile methods in software engineering education}
	Agile Conference, 2009. AGILE 09

\bibitem[RFC791]{rfc791}
	{Internet Protocol}
	URL: \url{http://www.rfc-editor.org/rfc/rfc791.txt}, Zugriff: M�rz 2014

\bibitem[RFC792]{rfc792}
	{INTERNET CONTROL MESSAGE PROTOCOL}
	URL: \url{http://www.rfc-editor.org/rfc/rfc792.txt}, Zugriff: M�rz 2014

\bibitem[RFC1055]{rfc1055}
	{A NONSTANDARD FOR TRANSMISSION OF IP DATAGRAMS OVER SERIAL LINES: SLIP}
	URL: \url{http://www.rfc-editor.org/rfc/rfc1055.txt}, Zugriff: M�rz 2014

\bibitem[RFC1071]{rfc1071}
	{Computing the Internet Checksum}
	URL: \url{http://www.rfc-editor.org/rfc/rfc1071.txt}, Zugriff: M�rz 2014

\bibitem[RFC1349]{rfc1349}
	{Type of Service in the Internet Protocol Suite}
	URL: \url{http://www.rfc-editor.org/rfc/rfc1349.txt}, Zugriff: M�rz 2014

\bibitem[RFC1700]{rfc1700}
	{Assigned Numbers}
	URL: \url{http://www.rfc-editor.org/rfc/rfc1700.txt}, Zugriff: M�rz 2014

\bibitem[RFC2460]{rfc2460}
	{Internet Protocol, Version 6 (IPv6)}
	URL: \url{http://www.rfc-editor.org/rfc/rfc2460.txt}, Zugriff: M�rz 2014

%\bibitem[SEM95]{sem95}
%	{National Semiconductor},
%	{\em PC16550D Universal Asynchronous Receiver Transmitter with FIFOs}
%	National Semiconductor, Inc, 1997

\bibitem[SK01]{sk01}
	{Britta Schinzel, Karin Kleinn},
	{\em Quo vadis, Informatik?}
	Informatik Spektrum, Volume 24, 2001, S. 91-97

\bibitem[SPO98]{spo98}
	{Robert Spotnitz},
	{\em Literate programming}
	Computers Chem. Engng, Volume 22, 1998, S. 1745-1747

\bibitem[SWE14]{swe14}
	{Sweave Homepage},
	URL: \url{http://www.stat.uni-muenchen.de/~leisch/Sweave/}, Zugriff: April 2014

\bibitem[THI01]{thi12}
	{Sebastien Li-Thiao-Te},
	{\em Literate Program Execution for Reproducible Research and Executable Papers}
	Procedia Computer Science, 9, 2012, S. 439-448

\bibitem[TW97]{tw97}
	{Andrew S. Tanenbaum, Albert S. Woodhull},
	{\em Operating Systems - Design and Implementation}
	Prentice-Hall Internation, Inc, 1997

\bibitem[UIP14]{uip14}
	{\textmu IP - Micro IP},
	URL: \url{http://www.sics.se/~adam/uip/}, Zugriff: M�rz 2014
	
\bibitem[VBOX14]{vbox14}
	{VirtualBox},
	URL: \url{www.virtualbox.org}, Zugriff: April 2014

\bibitem[VMSG14]{vmsg14}
	{Raoul Vallon, Michael M�ller-Wernhart, Wolfgang Schramm, Thomas Grechenig},
	{\em Kombination von Agil und Lean in der Softwareentwicklung }
	Informatik Spektrum, Volume 37, 2014, S. 28 - 35

\bibitem[XV14]{xv14}
	{vx6 - a simple Unix-like teaching operating system},
	URL: \url{http://pdos.csail.mit.edu/6.828/2012/xv6.html}, Zugriff: April 2014
			
\bibitem[YUN91]{yun91}
	{Donald E. Knuth},
	{\em Literate Programming System CDS}
	Journal of Computer Science \& Technologie, Volume 6, Nr. 3, 1991, S. 263-270
		
\bibitem[WIR14]{wir14}
	{Wireshark},
	URL: \url{www.wireshark.org}, Zugriff: April 2014
		
		
%%% ---- VORLAGEN ----
%\bibitem[]{}
%	{  }, 
%	{\em  }
		

\end{thebibliography}


\end{document}
